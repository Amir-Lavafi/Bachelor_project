\chapter{راهکارهای درون‌زنجیره‌ای: میکسرها و کوین‌جوین}
\label{chap:onchain_solutions}

در فصل پیشین، به تفصیل نشان دادیم که چگونه شفافیت ذاتی بلاکچین بیت‌کوین، همراه با تکنیک‌های پیشرفته تحلیل زنجیره، سپر شبه‌ناشناسی را در هم می‌شکند. این آسیب‌پذیری‌ها نیاز به ابزارهای فعالانه برای محافظت از حریم خصوصی را آشکار می‌سازد.

در این فصل، به بررسی اولین دسته از این ابزارها می‌پردازیم: «راهکارهای درون‌زنجیره‌ای»\LTRfootnote{On-Chain}. این راهکارها تلاش می‌کنند تا با ایجاد ابهام و شکستن پیوندهای قطعی در گراف تراکنش‌ها، مستقیماً بر روی خود بلاکچین بیت‌کوین، حریم خصوصی را تقویت کنند. هدف اصلی این تکنیک‌ها, افزایش «انکارپذیری قابل قبول»\LTRfootnote{Plausible Deniability} است؛ به این معنا که یک ناظر خارجی نتواند با اطمینان صددرصدی، یک تراکنش خاص را به یک کاربر مشخص نسبت دهد.

\section{فلسفه ابهام‌زایی و مجموعه ناشناسی}

حریم خصوصی در یک سیستم شفاف مانند بیت‌کوین، به معنای رمزنگاری یا پنهان کردن داده‌ها نیست. در عوض، حریم خصوصی از طریق «ابهام‌زایی»\LTRfootnote{Obfuscation} به دست می‌آید. ایده اصلی این است که در میان جمعیت پنهان شوید. هرچه جمعیتی که خود را در آن پنهان می‌کنید بزرگ‌تر باشد، حریم خصوصی شما نیز قوی‌تر خواهد بود.

این مفهوم در رمزنگاری با عنوان «مجموعه ناشناسی»\LTRfootnote{Anonymity Set} شناخته می‌شود \cite{narayanan_deanonymizing}. مجموعه ناشناسی به تعداد کل شرکت‌کنندگان مستقلی اشاره دارد که یک ناظر خارجی نمی‌تواند آن‌ها را از یکدیگر تشخیص دهد. هدف تمام تکنیک‌های این فصل، افزایش هرچه بیشتر اندازه این مجموعه است.

\section{میکسرهای متمرکز: راهکاری مبتنی بر اعتماد}

اولین تلاش‌ها برای افزایش حریم خصوصی در بیت‌کوین، از طریق سرویس‌هایی به نام «میکسرهای متمرکز»\LTRfootnote{Centralized Mixers} یا «تامبلرها»\LTRfootnote{Tumblers} انجام شد. سازوکار این سرویس‌ها ساده است: کاربر کوین‌های خود را به سرویس ارسال می‌کند، سرویس آن‌ها را با کوین‌های دیگران مخلوط کرده و معادل آن را از آدرس‌های دیگر به مقصدی جدید بازمی‌گرداند.

\subsection{نقاط ضعف حیاتی میکسرهای متمرکز}
با وجود سادگی ظاهری، این روش دارای نقاط ضعف ساختاری و خطرناکی است:
\begin{itemize}
	\item \textbf{ریسک اعتماد و سرقت:} این سرویس‌ها کاملاً «حضانتی»\LTRfootnote{Custodial} هستند. شما باید دارایی خود را مستقیماً به کنترل شخص ثالثی بسپارید و اعتماد کنید که او پول شما را پس خواهد داد.
	
	\item \textbf{ریسک ثبت گزارش:}\LTRfootnote{Logging} اپراتور میکسر به طور کامل از ارتباط بین آدرس‌های ورودی و خروجی مطلع است. این گزارش‌ها می‌توانند به سرقت رفته یا در اختیار نهادهای قانونی قرار گیرند و حریم خصوصی را به طور کامل از بین ببرند.
	
	\item \textbf{انگشت‌نگاری در تحلیل زنجیره:} تراکنش‌های ورودی و خروجی به میکسرها به راحتی قابل شناسایی هستند. این موضوع به خودی خود می‌تواند یک «پرچم قرمز»\LTRfootnote{Red flag} باشد و باعث شود که این کوین‌ها توسط صرافی‌ها به عنوان دارایی «پرخطر» شناسایی شوند.
\end{itemize}
این نقاط ضعف اساسی، جامعه را به سمت راهکاری بهتر و غیرمتمرکز سوق داد: کوین‌جوین.

\section{کوین‌جوین: انقلابی غیرمتمرکز و مشارکتی}

«کوین‌جوین»\LTRfootnote{CoinJoin} یک پروتکل رمزنگاری است که به چندین کاربر اجازه می‌دهد تا بدون نیاز به اعتماد به یکدیگر، یک تراکنش واحد و مشترک بسازند \cite{maxwell_coinjoin}. این راهکار، پاسخی مستقیم به ضعف‌های میکسرهای متمرکز است، زیرا «غیرحضانتی»\LTRfootnote{Non-Custodial} است و کاربران کنترل دارایی خود را حفظ می‌کنند.

ایده اصلی کوین‌جوین، شکستن مستقیم «فرض مالکیت مشترک ورودی‌ها» است. یک تراکنش کوین‌جوین به صورت هدفمند، ورودی‌هایی از چندین کاربر مستقل را در خود جای می‌دهد. وقتی یک تحلیلگر به چنین تراکنشی نگاه می‌کند، دیگر نمی‌تواند از روش‌های اکتشافی خود برای خوشه‌بندی آدرس‌ها استفاده کند.

\subsection{پیاده‌سازی‌ها و تکامل کوین‌جوین}
پروتکل کوین‌جوین در طول سال‌ها تکامل یافته است. چارچوب نظری مهمی مانند «زیرولینک»\LTRfootnote{ZeroLink} توسط \lr{Adam Fiscor} ارائه شد که پایه و اساس کیف پول‌های مدرن را تشکیل داد.
\begin{itemize}
	\item \textbf{کیف پول واسابی:}\LTRfootnote{Wasabi Wallet} یکی از مشهورترین پیاده‌سازی‌هاست که در آن یک «هماهنگ‌کننده»\LTRfootnote{Coordinator} به ساخت تراکنش کمک می‌کند اما هرگز به کلیدهای خصوصی کاربران دسترسی ندارد. ویژگی کلیدی آن، ایجاد خروجی‌هایی با مقادیر یکسان است که مجموعه ناشناسی را به شدت افزایش می‌دهد \cite{antonopoulos_mastering}.
	
	\item \textbf{کیف پول سامورایی و ویرپول:}\LTRfootnote{Samourai Wallet and Whirlpool} این کیف پول از استخرهای میکس همیشه فعال با مقادیر ثابت استفاده می‌کند. مزیت اصلی آن، تمرکز بر ابزارهای «پس از میکس»\LTRfootnote{Post-Mix} مانند \lr{Stonewall} است که به حفظ حریم خصوصی در هنگام خرج کردن کوین‌های میکس‌شده کمک می‌کند.
\end{itemize}

\section{محدودیت‌ها و حملات علیه کوین‌جوین}
با وجود قدرت بالای کوین‌جوین، این تکنیک نیز بی‌نقص نیست:
\begin{itemize}
	\item \textbf{تحلیل زمانی:}\LTRfootnote{Timing Analysis} یک مهاجم که در دور میکس شرکت می‌کند، ممکن است با تحلیل زمان ورود و خروج کاربران، اطلاعاتی را استنتاج کند.
	
	\item \textbf{مشکل آلودگی:}\LTRfootnote{Taint Problem} برخی صرافی‌ها ممکن است کوین‌هایی را که از یک تراکنش کوین‌جوین عبور کرده‌اند، به عنوان «مشکوک» پرچم‌گذاری کنند \cite{antonopoulos_mastering}.
	
	\item \textbf{حملات محروم‌سازی از سرویس:}\LTRfootnote{Denial-of-Service (DoS)} یک شرکت‌کننده مخرب می‌تواند با عدم امضای تراکنش نهایی، کل فرآیند را برای دیگران مختل کند.
\end{itemize}

در نهایت، کوین‌جوین یک گام بزرگ رو به جلو در جهت بازیابی حریم خصوصی در یک دفتر کل شفاف است. این تکنیک هسته‌ی اصلی ابزارهای تحلیل بلاکچین را به چالش می‌کشد، اما حفظ حریم خصوصی همچنان یک فرآیند فعال و نیازمند دانش است.