\chapter*{نتیجه‌گیری و جمع‌بندی}
\addcontentsline{toc}{chapter}{نتیجه‌گیری و جمع‌بندی}
\label{chap:conclusion}

این گزارش با هدف بررسی یکی از پیچیده‌ترین و حیاتی‌ترین جنبه‌های فناوری بلاکچین، یعنی «حریم خصوصی»، آغاز شد. ما سفر خود را با رد کردن تصور غلط رایج مبنی بر «ناشناسی» مطلق در بیت‌کوین آغاز کردیم و نشان دادیم که ماهیت «شبه‌ناشناس» و دفتر کل عمومی آن، چگونه بستری برای ردیابی و تحلیل فراهم می‌کند. این گزارش، مسیر حرکت از درک «مشکل» به سمت تحلیل «راهکارها» و در نهایت، مقایسه «فلسفه‌ها»ی متفاوت در این حوزه بود.

در فصل اول، پایه‌های فنی آسیب‌پذیری حریم خصوصی در بیت‌کوین را بنا نهادیم و نشان دادیم که مدل شفاف «\lr{UTXO}» و روش‌های اکتشافی مانند «خوشه‌بندی آدرس»، چگونه به شرکت‌های تحلیل بلاکچین اجازه می‌دهد تا هویت کاربران را کشف کنند \cite{narayanan_deanonymizing}.

در پاسخ به این چالش‌ها، دو دسته اصلی از راهکارها را بررسی کردیم. فصل دوم به «راهکارهای درون‌زنجیره‌ای» اختصاص یافت و نشان داد که پروتکل «کوین‌جوین»\LTRfootnote{CoinJoin} چگونه با ایجاد تراکنش‌های مشارکتی، اصل «مالکیت مشترک ورودی‌ها» را نقض کرده و ابهام را به زنجیره تزریق می‌کند \cite{maxwell_coinjoin}.

فصل سوم، پارادایم را تغییر داد و به «راهکارهای برون‌زنجیره‌ای» پرداخت. «شبکه لایتنینگ»\LTRfootnote{Lightning Network}، به عنوان یک راهکار لایه دوم، نشان داد که حریم خصوصی می‌تواند محصول جانبی ارزشمندِ تلاش برای «مقیاس‌پذیری» باشد \cite{decker_sok}. با این حال، این فصل همچنین نشان داد که این رویکرد، خود مجموعه‌ی جدیدی از چالش‌ها مانند «تحلیل ترافیک شبکه» را به همراه دارد.

در نهایت، در فصل چهارم، به مطالعه موردی «مونرو»\LTRfootnote{Monero} پرداختیم. مونرو نماینده یک فلسفه کاملاً متفاوت است: «حریم خصوصی به عنوان پیش‌فرض». این پروژه با اجباری کردن تکنیک‌های پیشرفته، تعویض‌پذیری کامل را تضمین می‌کند اما به قیمت کاهش مقیاس‌پذیری تمام می‌شود \cite{monero_website}.

\section*{نگاه به آینده}

سفری که در این گزارش طی شد، به یک نتیجه‌گیری واضح می‌رسد: هیچ راه حل جادویی یا «گلوله نقره‌ای»\LTRfootnote{Silver Bullet} برای مسئله حریم خصوصی در بلاکچین وجود ندارد. هر یک از رویکردهای بررسی‌شده مجموعه‌ای از بده‌بستان‌های خاص خود را در زمینه کارایی، هزینه، مقیاس‌پذیری و مقبولیت قانونی به همراه دارد.

\begin{itemize}
	\item \textbf{بیت‌کوین} به سمت یک لایه پایه شفاف و امن حرکت می‌کند که حریم خصوصی می‌تواند به صورت اختیاری و در لایه‌های بالاتر به آن افزوده شود.
	\item \textbf{مونرو} نشان‌دهنده یک سیستم مالی است که در آن حریم خصوصی یک حق غیرقابل انکار و پیش‌فرض است، حتی اگر به قیمت کارایی کمتر تمام شود.
\end{itemize}

آینده حریم خصوصی در فضای ارزهای دیجیتال، احتمالاً در گرو یک «بازی موش و گربه» دائمی بین توسعه‌دهندگان ابزارهای حریم خصوصی و شرکت‌های تحلیل بلاکچین خواهد بود (برای مثال، شرکت‌هایی مانند «سایفرین»\LTRfootnote{Cyfrin} به طور مداوم در حال بررسی امنیت این پروتکل‌ها هستند \cite{cyfrin_website}). با پیشرفت تکنیک‌های تحلیل، ابزارهای حفظ حریم خصوصی نیز پیچیده‌تر خواهند شد.

در نهایت، می‌توان گفت که حریم خصوصی در بلاکچین یک مفهوم صفر و یک نیست، بلکه یک «طیف» است. ابزارها و پلتفرم‌های مختلف به کاربران اجازه می‌دهند تا بسته به نیاز، مدل تهدید و اولویت‌های خود، جایگاهشان را در این طیف انتخاب کنند. این انتخاب، صرفاً یک تصمیم فنی نیست، بلکه بازتابی از دیدگاه فرد نسبت به ماهیت پول، آزادی و نظارت در عصر دیجیتال است.