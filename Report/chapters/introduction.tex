\chapter{مقدمه}
\label{chap:introduction}

بلاکچین، فناوری زیربنایی ارزهای دیجیتالی مانند بیت‌کوین، اغلب با کلیدواژه‌هایی مانند «شفافیت»، «تغییرناپذیری» و «ناشناسی»\LTRfootnote{Anonymous} توصیف می‌شود. در حالی که دو ویژگی اول تا حد زیادی صحیح هستند، ویژگی سوم، یعنی ناشناسی، یکی از بزرگ‌ترین و رایج‌ترین تصورات غلط در مورد این فناوری است. در حقیقت، اکثر بلاکچین‌های عمومی مانند بیت‌کوین، ناشناس نیستند، بلکه «شبه‌ناشناس»\LTRfootnote{Pseudonymous} هستند \cite{antonopoulos_mastering}. این تمایز ظریف، نقطه آغازین بحث پیچیده و حیاتی «حریم خصوصی در بلاکچین» است.

در این گزارش، ما به بررسی عمیق مفهوم حریم خصوصی در بستر بلاکچین، با تمرکز ویژه بر شبکه بیت‌کوین، خواهیم پرداخت. ابتدا به این پرسش پاسخ می‌دهیم که حریم خصوصی در این اکوسیستم دیجیتال به چه معناست، سپس اهمیت آن را در ابعاد مختلف تحلیل کرده و در نهایت، به چالش‌ها و خطراتی می‌پردازیم که نقض حریم خصوصی می‌تواند برای کاربران به همراه داشته باشد.

\section{حریم خصوصی در بلاکچین چیست؟}

بر خلاف سیستم‌های مالی سنتی، در بلاکچین بیت‌کوین تمام تراکنش‌ها در یک «دفتر کل عمومی و توزیع‌شده»\LTRfootnote{Public Ledger} ثبت و برای همگان قابل مشاهده است \cite{nakamoto_whitepaper}. هر کسی می‌تواند مقدار بیت‌کوین منتقل‌شده، آدرس‌های فرستنده و گیرنده، و زمان دقیق هر تراکنش را از ابتدای پیدایش شبکه تا به امروز مشاهده کند.

در این سیستم، هویت کاربران پشت آدرس‌های رمزنگاری‌شده پنهان می‌شود. این آدرس‌ها رشته‌هایی طولانی از حروف و اعداد هستند (مانند \lr{\texttt{1A1zP1eP5QGefi2DMPTfTL5SLmv7DivfNa}}) که هیچ اطلاعات هویتی مستقیمی از صاحب خود را فاش نمی‌کنند. این همان «شبه‌ناشناسی» است: تراکنش‌های شما عمومی هستند، اما هویت واقعی شما (در حالت ایده‌آل) خصوصی باقی می‌ماند. حریم خصوصی در این بستر، به معنای «توانایی یک کاربر برای انجام تراکنش بدون آنکه دیگران بتوانند آن تراکنش را به هویت واقعی او در دنیای فیزیکی مرتبط کنند» تعریف می‌شود.

\section{چرا حریم خصوصی برای ما مهم است؟}

اهمیت حریم خصوصی در یک سیستم مالی شفاف مانند بلاکچین، فراتر از پنهان کردن فعالیت‌های غیرقانونی است و جنبه‌های حیاتی برای آزادی فردی و کارایی اقتصادی دارد.

\begin{itemize}
	\item \textbf{امنیت شخصی و مالی:} اگر تمام دارایی و تاریخچه تراکنش‌های یک فرد برای عموم قابل مشاهده باشد، آن فرد به یک هدف برای سارقان، هکرها و باج‌گیران تبدیل می‌شود. دانستن اینکه یک آدرس خاص حاوی میلیون‌ها دلار بیت‌کوین است، انگیزه‌ای قوی برای حمله به صاحب آن آدرس ایجاد می‌کند.
	
	\item \textbf{حفظ اطلاعات تجاری:} شرکت‌ها و کسب‌وکارها نمی‌توانند اطلاعات مالی خود، مانند حقوق کارمندان، حجم معاملات با تأمین‌کنندگان، و استراتژی‌های سرمایه‌گذاری را به صورت عمومی فاش کنند. شفافیت کامل، مزیت رقابتی آن‌ها را از بین می‌برد.
	
	\item \textbf{مقاومت در برابر سانسور و کنترل:} در کشورهایی با نظام‌های سیاسی سرکوبگر، حریم خصوصی مالی به مخالفان سیاسی، روزنامه‌نگاران و فعالان حقوق بشر اجازه می‌دهد تا بدون ترس از مسدود شدن حساب‌ها یا ردیابی توسط دولت، به فعالیت‌های خود ادامه دهند.
	
	\item \textbf{تعویض‌پذیری:}\LTRfootnote{Fungibility} این یک مفهوم کلیدی اقتصادی است که می‌گوید هر واحد از یک کالا یا پول باید با واحد دیگری از همان نوع قابل تعویض و دارای ارزش یکسان باشد \cite{antonopoulos_mastering}. اما در بیت‌کوین، اگر تاریخچه یک کوین نشان دهد که در گذشته در یک فعالیت غیرقانونی استفاده شده، ممکن است برخی صرافی‌ها از پذیرش آن خودداری کنند. این «کوین‌های لکه‌دار» ارزش کمتری نسبت به «کوین‌های تمیز» پیدا می‌کنند و خاصیت تعویض‌پذیری بیت‌کوین را تضعیف می‌کنند.
\end{itemize}

\section{خطرات و چالش‌های حریم خصوصی ناکافی}

همانطور که اشاره شد، شبه‌ناشناسی بیت‌کوین یک سپر شکننده است. «شرکت‌های تحلیل بلاکچین»\LTRfootnote{Blockchain Analysis Firms} با استفاده از تکنیک‌های پیشرفته داده‌کاوی، می‌توانند الگوهای رفتاری کاربران را شناسایی کرده و آدرس‌های مختلف را به یک نهاد واحد متصل کنند (فرآیندی به نام «تحلیل خوشه‌ای»\LTRfootnote{Clustering}). این شرکت‌ها با تحلیل گراف تراکنش‌ها و ترکیب آن با «اطلاعات خارج از زنجیره»\LTRfootnote{Off-chain Data}، مانند اطلاعات «احراز هویت»\LTRfootnote{Know Your Customer (KYC)} از صرافی‌ها، می‌توانند هویت واقعی صاحبان آدرس‌ها را با دقت بالایی کشف کنند \cite{narayanan_deanonymizing}.

این فرآیند که «هویت‌زدایی»\LTRfootnote{Deanonymization} نام دارد، خطرات جدی به همراه دارد:
\begin{enumerate}
	\item \textbf{نظارت گسترده:} دولت‌ها و شرکت‌های بزرگ می‌توانند تمام فعالیت‌های مالی افراد را زیر نظر بگیرند. این اطلاعات می‌تواند برای اهداف تبلیغاتی، کنترل اجتماعی یا حتی سرکوب سیاسی مورد استفاده قرار گیرد.
	
	\item \textbf{نقض حریم خصوصی گذشته:} از آنجایی که بلاکچین تغییرناپذیر است، یک اشتباه کوچک در مدیریت حریم خصوصی (مانند استفاده مجدد از یک آدرس) می‌تواند تمام تراکنش‌های گذشته و آینده شما را برای همیشه به هویت‌تان پیوند بزند.
	
	\item \textbf{تبعیض مالی:} اطلاعات مربوط به نحوه خرج کردن پول توسط افراد می‌تواند منجر به تبعیض شود. برای مثال، ممکن است یک شرکت بیمه بر اساس تاریخچه خرید شما، نرخ بالاتری را برایتان تعیین کند یا یک بانک بر اساس الگوی تراکنش‌هایتان، از ارائه وام به شما خودداری کند.
\end{enumerate}

بنابراین، حریم خصوصی در بلاکچین یک مفهوم صفر و یک نیست، بلکه یک طیف است. درک آسیب‌پذیری‌ها و ابزارهای موجود برای تقویت آن، برای هر کاربری که به دنبال حفظ حاکمیت مالی و امنیت خود در این دنیای جدید است، امری ضروری محسوب می‌شود.