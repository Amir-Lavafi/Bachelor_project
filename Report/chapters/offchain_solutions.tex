\chapter{راهکارهای برون‌زنجیره‌ای: شبکه لایتنینگ و پیامدهای آن}
\label{chap:offchain_solutions}

در فصل گذشته، راهکارهایی را بررسی کردیم که مستقیماً روی زنجیره اصلی بیت‌کوین عمل می‌کنند. با این حال، تمام این راهکارها در نهایت محدود به یک واقعیت اساسی هستند: تمام داده‌ها باید روی یک دفتر کل عمومی ثبت شوند. این محدودیت، انگیزه اصلی برای توسعه‌ی دسته‌ی کاملاً متفauti از راهکارها بود: «راهکارهای برون‌زنجیره‌ای»\LTRfootnote{Off-Chain}.

این راهکارها در وهله اول برای حل مشکل «مقیاس‌پذیری»\LTRfootnote{Scalability} بیت‌کوین طراحی شدند، اما به عنوان یک محصول جانبی، مزایای بسیار قابل توجهی برای حریم خصوصی به ارمغان آوردند \cite{decker_sok}. ایده اصلی این است: «بهترین راه برای پنهان کردن یک تراکنش از تحلیلگران، این است که آن تراکنش هرگز روی زنجیره ثبت نشود.» در این فصل، به بررسی مهم‌ترین راهکار لایه دوم بیت‌کوین، یعنی «شبکه لایتنینگ»\LTRfootnote{Lightning Network}، می‌پردازیم.

\section{معرفی شبکه لایتنینگ و سازوکار آن}

شبکه بیت‌کوین به صورت ذاتی کند و پرهزینه است. این ساختار، آن را برای پرداخت‌های خرد و روزمره ناکارآمد می‌سازد. شبکه لایتنینگ یک «پروتکل لایه دوم»\LTRfootnote{Layer 2 Protocol} است که بر بستر بیت‌کوین ساخته شده تا این مشکل را حل کند. سازوکار اصلی این شبکه بر اساس «کانال‌های پرداخت»\LTRfootnote{Payment Channels} و «قراردادهای هوشمند زمان‌بندی‌شده مبتنی بر هش»\LTRfootnote{Hashed Time Locked Contracts (HTLC)} استوار است \cite{antonopoulos_mastering}.

\subsection{کانال‌های پرداخت و قراردادهای \lr{HTLC}}
یک کانال پرداخت، یک قرارداد دوجانبه بین دو کاربر است که با یک تراکنش روی زنجیره اصلی باز می‌شود. پس از آن، کاربران می‌توانند بی‌نهایت تراکنش فوری و ارزان را خارج از زنجیره با یکدیگر انجام دهند. قدرت واقعی شبکه زمانی آشکار می‌شود که این کانال‌ها به یکدیگر متصل می‌شوند. پرداخت‌ها بین کاربرانی که کانال مستقیم ندارند، از طریق قراردادهای \lr{HTLC} به صورت امن و بدون نیاز به اعتماد مسیریابی می‌شوند.
یک \lr{HTLC} تضمین می‌کند که یک پرداخت یا به صورت کامل و زنجیره‌ای تا مقصد نهایی انجام می‌شود، یا در صورت بروز مشکل، پول پس از یک «قفل زمانی»\LTRfootnote{Time Lock} مشخص به فرستنده اصلی بازمی‌گردد. این سازوکار از سرقت وجوه توسط نودهای میانی جلوگیری می‌کند.

\section{مزایای حریم خصوصی در شبکه لایتنینگ}

ساختار لایتنینگ به طور طبیعی چندین مزیت مهم برای حریم خصوصی ایجاد می‌کند \cite{decker_sok}:
\begin{itemize}
	\item \textbf{کاهش چشمگیر ردپای روی زنجیره:} به جای ثبت تمام تراکنش‌ها، تنها دو تراکنش (افتتاح و بستن کانال) روی بلاکچین ثبت می‌شود. این کار، داده‌های موجود برای تحلیلگران را به شدت کاهش می‌دهد.
	
	\item \textbf{محرمانگی جزئیات تراکنش:} جزئیات هر پرداخت (مقدار، زمان و طرفین) به صورت عمومی پخش نمی‌شود و تنها برای شرکت‌کنندگان در مسیر پرداخت قابل مشاهده است.
	
	\item \textbf{مسیریابی پیازی:}\LTRfootnote{Onion Routing} شبکه لایتنینگ از تکنیکی مشابه شبکه «تور»\LTRfootnote{Tor} به نام «مسیریابی پیازی» استفاده می‌کند. هر نود میانی در مسیر پرداخت، تنها گره قبلی و بعدی خود را می‌شناسد و از مبدا و مقصد نهایی بی‌اطلاع است. این سازوکار، ردیابی کامل مسیر پرداخت را دشوار می‌سازد.
\end{itemize}

\section{محدودیت‌ها و بردارهای حمله جدید}

با وجود مزایای ذکر شده، لایتنینگ یک راه حل بی‌نقص نیست و مجموعه‌ی جدیدی از چالش‌ها را در سطح شبکه معرفی می‌کند \cite{decker_sok}.

\subsection{تحلیل در سطح شبکه}
یک مهاجم می‌تواند با اجرای نودهای بزرگ یا «هاب»\LTRfootnote{Hub}، حجم زیادی از ترافیک را مشاهده کرده و با تحلیل‌های آماری، اطلاعاتی را استخراج کند.
\begin{itemize}
	\item \textbf{تحلیل زمانی:}\LTRfootnote{Timing Analysis} یک حمله کلاسیک که در آن مهاجم با کنترل چند نود، سعی می‌کند با تطبیق زمان و مقدار پرداخت‌های ورودی و خروجی، مسیرها را به هم مرتبط کند.
	\item \textbf{کاوش کانال:}\LTRfootnote{Channel Probing} یک حمله فعال که در آن مهاجم با ارسال پرداخت‌های ساختگی ناموفق، اطلاعاتی در مورد موجودی کانال‌های دیگران به دست می‌آورد.
	\item \textbf{حمله ازدحام کانال:}\LTRfootnote{Channel Jamming Attack} یک حمله پیشرفته از نوع «محروم‌سازی از سرویس»\LTRfootnote{Denial-of-Service (DoS)} که در آن مهاجم با ارسال پرداخت‌های معلق، نقدینگی کانال‌ها را قفل کرده و آن‌ها را غیرقابل استفاده می‌کند.
\end{itemize}

\subsection{آسیب‌پذیری‌های مرتبط با زنجیره اصلی}
اتصال این شبکه به زنجیره اصلی، خود منبعی برای نشت اطلاعات است. تراکنش‌های عمومی افتتاح و بستن کانال، «گراف کانال‌های» شبکه را عمومی کرده و کاربران فعال لایتنینگ را قابل شناسایی می‌کنند. همچنین، نیاز به سرویس‌های «برج مراقبت»\LTRfootnote{Watchtowers} برای جلوگیری از تقلب، یک بده‌بستان حریم خصوصی ایجاد می‌کند، زیرا این سرویس‌ها از جزئیات کانال شما مطلع می‌شوند.

\subsection{نتیجه‌گیری: یک بده‌بستان جدید}
شبکه لایتنینگ مدل حریم خصوصی در بیت‌کوین را به طور بنیادین تغییر می‌دهد. این شبکه با حذف جزئیات تراکنش‌های فردی از دفتر کل عمومی، یک پیروزی بزرگ برای حریم خصوصی روزمره محسوب می‌شود. با این حال، این راهکار، کاربران را وارد یک محیط شبکه‌ای پویا و پیچیده می‌کند که در آن بردارهای حمله جدیدی ظهور می‌کنند.