\chapter{تحلیل زنجیره و آسیب‌پذیری‌های حریم خصوصی در بیت‌کوین}
\label{chap:analysis_vulnerabilities}

در فصل مقدمه، به این حقیقت پرداختیم که بیت‌کوین برخلاف تصور عمومی، ناشناس نیست، بلکه «شبه‌ناشناس»\LTRfootnote{Pseudonymous} است. این شبه‌ناشناسی بر این فرض استوار است که ارتباطی بین آدرس‌های روی بلاکچین و هویت‌های واقعی در دنیای خارج وجود ندارد. اما این فرض، یک سپر بسیار شکننده است. در این فصل، به صورت فنی و دقیق بررسی خواهیم کرد که این سپر چگونه شکسته می‌شود. ما به تحلیل ساختار داده‌ای بیت‌کوین می‌پردازیم و تکنیک‌هایی را که توسط شرکت‌های تحلیل بلاکچین برای هویت‌زدایی کاربران به کار می‌رود، تشریح می‌کنیم. درک عمیق این آسیب‌پذیری‌ها، پیش‌نیاز درک اهمیت راهکارهایی است که در فصول بعدی به آن‌ها خواهیم پرداخت.

\section{مدل خروجی خرج‌نشده تراکنش}

برای درک چگونگی ردیابی تراکنش‌ها در بیت‌کوین، ابتدا باید مدل حسابداری آن را بشناسیم. بیت‌کوین از «مدل حساب و موجودی»\LTRfootnote{Account/Balance Model} که در بانکداری سنتی یا بلاکچین‌هایی مانند اتریوم رایج است، استفاده نمی‌کند.

در مقابل، بیت‌کوین بر اساس مدل «خروجی خرج‌نشده تراکنش»\LTRfootnote{Unspent Transaction Output (UTXO)} کار می‌کند \cite{antonopoulos_mastering}. این مدل شباهت زیادی به استفاده از پول نقد فیزیکی دارد. کیف پول شما حاوی یک موجودی کلی نیست، بلکه مجموعه‌ای از اسکناس‌ها و سکه‌های مجزا (همان \lr{UTXO}ها) است. برای درک بهتر، یک مثال را در نظر بگیرید:
فرض کنید شما در کیف پول خود دو \lr{UTXO} مجزا دارید. حال می‌خواهید یک کالایی را بخرید که ارزش آن از هر یک از \lr{UTXO}های شما بیشتر است. شما باید هر دو \lr{UTXO} را به عنوان «ورودی»\LTRfootnote{Input} تراکنش خود «خرج» کنید. تراکنش شما دو «خروجی»\LTRfootnote{Output} جدید ایجاد می‌کند: یکی برای فروشنده و دیگری که به عنوان «باقی‌مانده» به یک آدرس جدید تحت کنترل خودتان برمی‌گردد.

نکته کلیدی این است که هر تراکنش، \lr{UTXO}های قبلی را مصرف کرده و \lr{UTXO}های جدیدی را خلق می‌کند. این فرآیند یک زنجیره‌ی به هم پیوسته از تراکنش‌ها را از گذشته تا به امروز ایجاد می‌کند که کاملاً شفاف و قابل ردیابی است. این شفافیت رادیکال، سنگ بنای تحلیل بلاکچین است.

\section{استفاده مجدد از آدرس: پاشنه آشیل حریم خصوصی}

اگر مدل \lr{UTXO} سنگ بنای تحلیل است، «استفاده مجدد از آدرس»\LTRfootnote{Address Reuse} بزرگ‌ترین اشتباهی است که یک کاربر می‌تواند مرتکب شود \cite{antonopoulos_mastering}. یک آدرس بیت‌کوین از نظر فنی برای یک بار استفاده طراحی شده است. کیف پول‌های مدرن که از «استانداردهای سلسله‌مراتبی قطعی»\LTRfootnote{Hierarchical Deterministic (HD) wallets} پیروی می‌کنند، برای هر تراکنش ورودی یک آدرس کاملاً جدید ایجاد می‌کنند. با این حال، بسیاری از کاربران به دلایل سادگی، یک آدرس ثابت را برای دریافت تمام وجوه خود به دیگران اعلام می‌کنند.

این کار معادل این است که شما تمام درآمدهای خود از منابع مختلف را به یک حساب بانکی شفاف و عمومی واریز کنید. این عمل بلافاصله تمام این نهادها را به یکدیگر و به یک هویت واحد (شما) مرتبط می‌سازد. برای مثال، اگر از یک صرافی معتبر که در آن «احراز هویت»\LTRfootnote{Know Your Customer (KYC)} کرده‌اید، مقداری بیت‌کوین به آدرس \lr{A} خود منتقل کنید و سپس دوست شما نیز به همان آدرس \lr{A} پولی واریز کند، یک تحلیلگر خارجی می‌تواند به راحتی این دو تراکنش را به هویت واقعی شما پیوند بزند.

\section{روش‌های اکتشافی: هنر تبدیل داده به اطلاعات}

شرکت‌های تحلیل بلاکچین صرفاً به داده‌های خام نگاه نمی‌کنند؛ آن‌ها از الگوریتم‌ها و «روش‌های اکتشافی»\LTRfootnote{Heuristics} برای استخراج اطلاعات معنادار از این داده‌ها استفاده می‌کنند \cite{narayanan_deanonymizing}. یک روش اکتشافی، یک قانون سرانگشتی یا یک فرض منطقی است که اگرچه صحت آن ۱۰۰٪ تضمین‌شده نیست، اما در اکثر موارد درست عمل می‌کند.

مهم‌ترین روش اکتشافی در تحلیل بیت‌کوین، «فرض مالکیت مشترک ورودی‌ها»\LTRfootnote{Common-Input-Ownership Heuristic} است. این اصل بیان می‌کند که اگر یک تراکنش چندین ورودی (\lr{UTXO}) داشته باشد، به احتمال بسیار زیاد تمام آن ورودی‌ها توسط یک شخص یا یک کیف پول کنترل می‌شوند. با استفاده مکرر از این روش، تحلیلگران فرآیندی به نام «خوشه‌بندی آدرس»\LTRfootnote{Address Clustering} را انجام می‌دهند. آن‌ها آدرس‌های مختلف را در یک «خوشه» که نماینده یک نهاد واحد است، گروه‌بندی می‌کنند و پروفایل‌های مالی دقیقی از فعالیت‌های کاربران ایجاد می‌کنند.

\section{گراف تراکنش‌ها و شرکت‌های تحلیل بلاکچین}

با در اختیار داشتن این ابزارها، کل بلاکچین بیت‌کوین را می‌توان به عنوان یک «گراف تراکنش»\LTRfootnote{Transaction Graph} عظیم تصور کرد. در این گراف، آدرس‌ها به عنوان «گره‌ها»\LTRfootnote{Nodes} و تراکنش‌ها به عنوان «یال‌های جهت‌دار»\LTRfootnote{Edges} در نظر گرفته می‌شوند. شرکت‌های پیشرو در این حوزه، مانند \lr{Chainalysis}، \lr{Elliptic} و \lr{CipherTrace}، این گراف را ساخته و به طور مداوم آن را تحلیل می‌کنند.

کار اصلی این شرکت‌ها، غنی‌سازی این گراف با «داده‌های خارج از زنجیره»\LTRfootnote{Off-chain Data} است \cite{narayanan_deanonymizing}. آن‌ها نقاط اتصال دنیای دیجیتال و دنیای واقعی (مانند صرافی‌ها) را شناسایی می‌کنند. برای مثال، اگر یک آدرس در یک خوشه بزرگ با یک صرافی در تعامل باشد، آن صرافی هویت واقعی صاحب آدرس را می‌داند. در نتیجه، شرکت تحلیلی می‌تواند کل خوشه را به آن هویت واقعی برچسب بزند. این فرآیند، شبه‌ناشناسی بیت‌کوین را به صورت سیستماتیک از بین می‌برد و اینجاست که نیاز به راهکارهای فعالانه برای حفظ حریم خصوصی، که در فصل بعد به آن‌ها می‌پردازیم، آشکار می‌شود.