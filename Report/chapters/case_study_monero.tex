\chapter{مطالعه موردی: مونرو، بلاکچین متمرکز بر حریم خصوصی}
\label{chap:case_study_monero}

در فصول گذشته، به بررسی آسیب‌پذیری‌های حریم خصوصی در بیت‌کوین و راهکارهایی پرداختیم که به کاربران «اختیار» می‌دهند تا حریم خصوصی خود را بهبود بخشند. با این حال، ماهیت «اختیاری بودن حریم خصوصی»\LTRfootnote{Opt-in Privacy}، خود یک نقطه ضعف است؛ زیرا استفاده از ابزارهای حریم خصوصی می‌تواند یک رفتار مشکوک تلقی شود.

در این فصل، رویکردی کاملاً متفاوت را بررسی می‌کنیم: «حریم خصوصی به عنوان پیش‌فرض»\LTRfootnote{Privacy by Default}. ما به مطالعه موردی «مونرو»\LTRfootnote{Monero}، مشهورترین ارز دیجیتال متمرکز بر حریم خصوصی، می‌پردازیم \cite{monero_website}. برخلاف بیت‌کوین، مونرو بر این فلسفه بنا شده که حریم خصوصی نباید یک ویژگی جانبی، بلکه باید هسته‌ی اصلی یک سیستم پولی دیجیتال باشد. این فصل به تحلیل فناوری‌های کلیدی مونرو و مقایسه بده‌بستان‌های رویکرد آن با بیت‌کوین اختصاص دارد.

\section{فلسفه مونرو: حریم خصوصی به عنوان یک حق}

تفاوت بنیادین بین بیت‌کوین و مونرو در شفافیت دفتر کل آن‌ها نهفته است. دفتر کل بیت‌کوین یک کتاب باز است، اما دفتر کل مونرو یک کتاب رمزنگاری‌شده است. این طراحی، پیامدهای عمیقی دارد:
\begin{itemize}
	\item \textbf{مجموعه ناشناسی جهانی:} در مونرو، از آنجایی که تمام تراکنش‌ها خصوصی هستند، «مجموعه ناشناسی» شما شامل تمام کاربران شبکه می‌شود که بسیار قوی‌تر از مجموعه محدود شرکت‌کنندگان در یک دور کوین‌جوین است.
	
	\item \textbf{حل مشکل تعویض‌پذیری:}\LTRfootnote{Fungibility} در مونرو، تاریخچه هیچ کوینی قابل ردیابی نیست، بنابراین تمام واحدها کاملاً یکسان و قابل تعویض هستند و مشکل «کوین‌های لکه‌دار» وجود ندارد.
\end{itemize}

\section{هسته‌ی فنی مونرو: سه ستون حریم خصوصی}

حریم خصوصی در مونرو بر سه ستون اصلی استوار است که فرستنده، گیرنده و مبلغ تراکنش را پنهان می‌کند \cite{monero_website}.

\subsection{امضاهای حلقوی: پنهان کردن فرستنده}
مونرو از «امضاهای حلقوی» استفاده می‌کند. در این تکنیک، فرستنده تراکنش را به نمایندگی از یک «حلقه» از کاربران امضا می‌کند. این حلقه شامل خروجی واقعی فرستنده و چندین خروجی دیگر به عنوان «طعمه»\LTRfootnote{Decoys} است. یک ناظر خارجی می‌تواند تأیید کند که امضا توسط یکی از اعضای حلقه انجام شده، اما نمی‌تواند تشخیص دهد کدام یک از آن‌ها امضاکننده واقعی بوده است.

\subsection{آدرس‌های مخفی: پنهان کردن گیرنده}
برای جلوگیری از عمومی شدن آدرس گیرنده، مونرو از «آدرس‌های مخفی» استفاده می‌کند. در این روش، برای هر تراکنش، یک آدرس عمومی «یکبار مصرف» و منحصر به فرد به نمایندگی از گیرنده ایجاد می‌شود. تنها گیرنده واقعی، با استفاده از کلید خصوصی خود، می‌تواند بلاکچین را اسکن کرده و تشخیص دهد که کدام یک از این آدرس‌های بی‌شمار یکبار مصرف به او تعلق دارد.

\subsection{تراکنش‌های محرمانه حلقوی: پنهان کردن مبلغ}
این پروتکل با استفاده از اثبات‌های رمزنگاری پیشرفته (مانند «تعهدات پدرسن»\LTRfootnote{Pedersen Commitments})، مبالغ تراکنش‌ها را به طور کامل رمزنگاری می‌کند. با این حال، این سیستم همچنان به شبکه اجازه می‌دهد تا صحت ریاضی تراکنش را (برابری مجموع ورودی‌ها و خروجی‌ها) تأیید کند، بدون آنکه از مقادیر واقعی مطلع شود. نوآوری‌های بعدی مانند «بات‌پروف‌ها»\LTRfootnote{Bulletproofs} حجم این تراکنش‌ها را به شدت کاهش داده‌اند.

\section{مقایسه و بده‌بستان‌ها: مونرو در برابر بیت‌کوین}

رویکرد مونرو مجموعه‌ای از «بده‌بستان‌ها»\LTRfootnote{Trade-offs} را در مقایسه با بیت‌کوین ایجاد می‌کند:
\begin{itemize}
	\item \textbf{حریم خصوصی در برابر مقیاس‌پذیری:} مونرو حریم خصوصی بسیار قوی‌تری ارائه می‌دهد، اما این به قیمت تراکنش‌هایی با حجم بسیار بزرگ‌تر و فرآیند تأیید سنگین‌تر تمام می‌شود که مقیاس‌پذیری آن را به چالش می‌کشد \cite{antonopoulos_mastering}.
	
	\item \textbf{شفافیت و حسابرسی:} شفافیت بیت‌کوین برای کسب‌وکارها و نهادهای نظارتی یک مزیت است. ماهیت غیرشفاف مونرو، پذیرش آن توسط نهادهای قانون‌مند را دشوار می‌سازد، هرچند «شفافیت اختیاری» از طریق «کلیدهای مشاهده»\LTRfootnote{View Keys} در آن ممکن است.
	
	\item \textbf{اکوسیستم و پذیرش:} بیت‌کوین به دلیل قدمت خود، دارای اکوسیستم و نقدینگی بسیار بزرگ‌تری است.
\end{itemize}

\subsection{نتیجه‌گیری: دو فلسفه برای آینده پول}
مونرو نماینده یک فلسفه کاملاً متفاوت در مورد ماهیت پول دیجیتال است که در آن حریم خصوصی یک ضرورت غیرقابل حذف است. در مقابل، بیت‌کوین به سمت یک لایه پایه شفاف حرکت کرده که حریم خصوصی می‌تواند به صورت اختیاری در لایه‌های بالاتر به آن اضافه شود. انتخاب بین این دو مدل، یک مناظره ایدئولوژیک در مورد آینده پول است: آیا سیستم‌های مالی باید به صورت پیش‌فرض شفاف باشند یا خصوصی؟